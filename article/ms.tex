\documentclass[useAMS,usenatbib]{mn2e}

% Convenient colloquiualisms.
\newcommand\tc{\textit{The Cannon}}

% Convenient maths.
\newcommand\lv{\mathbf{l_n}}
\newcommand\cv{{\mathbf \theta}_\lambda}
\newcommand\given{|}

\title[Cannon Chemistry]{Detailed Chemical Abundances with \tc} 
% Some arrangement of the following authors:
\author[Casey et al.]{A.~R.~Casey$^1$, M.~Ness$^2$, G.~Gilmore$^1$,
    D.~W.~Hogg$^{2,3,4}$, H.~W.~Rix$^2$ \\ 
$^1$Institute of Astronomy, University of Cambridge, Madingley Road, Cambridge
    CB3 0HA, UK\\
$^2$Max-Planck-Institut f\"ur Astronomie, K\"onigstuhl 17, D-69117 Heidelberg,
    Germany\\
$^3$Center for Cosmology and Particle Physics, Department of Physics, New York
    University, 4 Washington Pl., room 424, New York, \\
    NY, 10003, USA\\
$^4$Center for Data Science, New York University, 726 Broadway, 7th Floor,
    New York, NY 10003, USA}
\begin{document}

\date{Accepted 2015 XX XX. Received 2015 October XX; in original form 2015 XX xx}

\pagerange{\pageref{firstpage}--\pageref{lastpage}} \pubyear{2015}

\maketitle

\label{firstpage}

\begin{abstract}
% large surveys spectroscopically planned.
% poses computing challenge.
% commonalityies between them: all R ~ 20,000, trying to get abundances from
% weak lines, which are linearly responsive (therefore best indicators) of
% abundance.
% TC provides a purely data-driven approach to stellar parameter (label)
% determination.
% fast, accurate, limited by systematics (which are low), meaning it works well
% in low S/N. Perfect for survey work.
% Simply extending dimensionality of the Cannon to include individual abundances
% can create enormous difficulties, both in potentially overfitting and curse of DIm.
% These extensions also imply a purely data-driven approach to abundances.
% Here we include radiative transfer physics to explain why a polynomial extension
% works so well, and provide a simple model to relate individual chemical
% abundance (and teh stellar parameter labels) to a flux value.
% We provide a direct link between model atmosphere physics and data-driven
% approaches, allowing the data to inform models.
% We demonstrate that individual abundances can be determined with a precision of
% X using our models. These determinations are fast: we estimate all 15 APOGEE 
% elemental abundances (and stellar parameters) for 60,000 stars on a modern
% laptop in 5 minutes.


\end{abstract}

\begin{keywords}
\end{keywords}

\section{Introduction}

% it's awesome: fast, accurate and precise (given the models published thus far)
% impressively it does all this whilst ignoring all advances made in stellar physics over the past 100 years.

\section{Model}


\tc{} provides a purely data-driven approach to stellar label determination. The
spectral model describes the flux at every pixel $f_{n\lambda}$ to be a function
of a coefficient vector $\cv$, and a label vector $\lv$:

\begin{equation}
    f_{n\lambda} = g(\lv\given\cv) + {\rm noise}.
\end{equation}

This form is flexible: the label vector $\lv$ may be linear in just a few labels
or a complex combination of all $K$ available labels. The underlying principle
is that the flux for any pixel from any star is predictable with a very simple
model (generally a polynomial), given the labels. Indeed, \citet{Ness2015}
employ a quadratic-in-labels (see their Equation 8) spectral model using stellar
parameters (effective temperature $T_{\rm eff}$, surface gravity $\log{g}$, and 
metallicity [M/H]) and demonstrate by cross-validation that these labels can be 
measured with a precision of 95~K in $T_{\rm eff}$, 0.24 in $\log{g}$, 0.08 in 
[M/H], with negligible biases.

Utilising a simple polynomial model to predict spectral fluxes is in stark
contrast to parallel efforts in radiative transfer of realistic (3D) stellar 
photospheres. Calculations of this nature are extremely expensive. Indeed, even
with simplications of these models (e.g., interpolated 1D photospheres),
calculating integrated flux requires knowledge of species opacities, detailed 
atomic and molecular information for all contributing elements, solving for
the equation of state, and performing radiative transfer calculations. Given the
simplicity of \tc{} and a quadratic-in-labels spectral model, it is remarkable
that these complex calculations can be simplified so much, whilst maintaining
the typical fidelity reported for high-resolution spectroscopic studies.




% existing simple model. replaces expensive synthesis, involving model atmosphere
% generation, t-tau relations, opacities, equation of state, integrated line-of-sight
% and radiative transfer, etc.

% before extending this model to include individual abundances, it is worthwhile
% demonstrating (from a physics perspective) *why* the cannon works so well.

% APOGEE, R of 20,000, 
% LAMOST, higher Rs
% super high-resolution also works.


\section{Analysis}
% Detailed chemical abundances with APOGEE.



\section{Conclusions}


\section*{Acknowledgments}
This research was fostered by discussions at the \textit{New Milky Way}
conference hosted by the Munich Institute for Astro- and Particle Physics, and
the \textit{Stellar Streams in the Local Universe} conference hosted at Ringberg
Castle by the Max-Planck-Institut F\"ur Astronomie. This research made use of 
Astropy, a community-developed core Python package for Astronomy \citep{astropy}.

\begin{thebibliography}{99}
%\bibitem[\protect\citeauthoryear{Baird}{1981}]{b1} Baird S.R., 1981,
%ApJ, 245, 208
\bibitem[Astropy Collaboration et al.(2013)]{astropy} Astropy Collaboration, Robitaille, T.~P., Tollerud, E.~J., et al.\ 2013, \aap, 558, AA33
\end{thebibliography}


\label{lastpage}

\end{document}

